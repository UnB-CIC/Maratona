%% FILE_NAME (gerado automaticamente em FILE_DATE)

\documentclass{UnBExam}%
\extraheadheight[0cm]{0mm}%
\documento{MaratonaUnB de Programação UnB}%
CONTEST_DATE

\makeatletter%
\newcommand{\infoHeader}{%
	\thispagestyle{empty}%

	\begin{center}%
		\includegraphics[width=.25\textwidth]{MaratonaUnB}%
	\end{center}%

	\begin{center}%
		{\Huge\bf\documento@UnBExam}%
		\\\vspace{1em}%
		{\small\emph{\data@UnBExam}}%
		\\\vspace{1em}%
		Informações.
	\end{center}%
}%
\newcommand{\infoFooter}{%
	\begin{center}\small v0.1\end{center}%
}%
\makeatother%

\begin{document}%
	\infoHeader%
	\vspace{2em}%
	\subsection*{Limites de Tempo}%
	Os valores são dados em segundos:\\%
	\begin{center}%
		\begin{tabular}[t]{c l LANGUAGE_COLUMNS}%
			\hline%
			Problema & Nome & LANGUAGE_NAMES \\\hline%
			PROBLEM_INFO
		\end{tabular}%
	\end{center}%

	\subsection*{Limites de Memória}%
	\begin{itemize}%
		\setlength{\itemsep}{0pt}%
		\setlength{\parskip}{0pt}%
		\item \texttt{C}, \texttt{C++}: 1\texttt{GB}%
		\item \texttt{Java}: 1\texttt{GB} + 20\texttt{MB} de pilha de execução%
	\end{itemize}%

	\subsection*{Outros Limites}%
	\begin{itemize}%
		\setlength{\itemsep}{0pt}%
		\setlength{\parskip}{0pt}%
		\item Tamanho do arquivo fonte: 100\texttt{KB}%
		\item Tamanho da saída: 1\texttt{MB}%
	\end{itemize}%

	\subsection*{Comandos de Compilação}%
	\begin{itemize}%
		\setlength{\itemsep}{0pt}%
		\setlength{\parskip}{0pt}%
		\item \texttt{C: gcc -static -O2 -lm}%
		\item \texttt{C++: g++ -static -O2 -lm}%
		\item \texttt{Java: javac}%
	\end{itemize}%

	\subsection*{\texttt{C/C++}}%
	\begin{itemize}%
		\setlength{\itemsep}{0pt}%
		\setlength{\parskip}{0pt}%
		\item Seu programa deve retornar \texttt{0} como último comando executado.
	\end{itemize}%

	\subsection*{\texttt{Java}}%
	\begin{itemize}%
		\setlength{\itemsep}{0pt}%
		\setlength{\parskip}{0pt}%
		\item Não declare um pacote em seu programa.%
		\item O nome do arquivo deve seguir a convenção, portanto o nome da sua classe pública deve ser uma letra maiúscula (A, B, $\cdots$, )
		\item Comando para executar uma solução: \texttt{java -Xms1024m -Xmx1024m -Xss20m}%
	\end{itemize}%
	% \subsection*{\texttt{Python}}%
% 	\begin{itemize}
% 		\item Be carefull to inform the correct version, using the extension .py2 for version 2 and the extension .py3 for version
% 3.
% 	\end{itemize}
	\vfill%
	\infoFooter
\end{document}%
