%% http://maratona.ime.usp.br/hist/2014/primeira-fase/

\NomeDoProblema{Mário}%
\Conceitos{Busca Binária}%
\Dificuldade{2}%

Mário é dono de uma empresa de guarda-volumes, a Armários a Custos Moderados (ACM). Mário conquistou sua clientela graças à rapidez no processo de armazenar os volumes. Para isso, ele tem duas técnicas:
\begin{itemize}
	\item Todos os armários estão dispostos numa fila e são numerados com inteiros positivos a partir de 1. Isso permite a Mário economizar tempo na hora de procurar um armário;
	\item Todos os armários têm rodinhas, o que lhe dá grande flexibilidade na hora de rearranjar seus armários (naturalmente, quando Mário troca dois armários de posição, ele também troca suas numerações, para que eles continuem numerados sequencialmente a partir de 1).
\end{itemize}

Para alugar armários para um novo cliente, Mário gosta de utilizar armários contíguos, pois no início da locação um novo cliente em geral faz muitas requisições para acessar o conteúdo armazenado,
e o fato de os armários estarem contíguos facilita o acesso para o cliente e para Mário.

Desde que Mário tenha armários livres em quantidade suficiente, ele sempre pode conseguir isso. Por exemplo, se a requisição de um novo cliente necessita de quatro armários, mas apenas os armários de número 1, 3, 5, 6, 8 estiverem disponíveis, Mário pode trocar os armários 5 e 2 e os armários 6 e 4 de posição: assim, ele pode alugar o intervalo de armários de 1 até 4.

No entanto, para minimizar o tempo de atendimento a um novo cliente, Mário quer fazer o menor número de trocas possível para armazenar cada volume. No exemplo acima, ele poderia simplesmente trocar os armários 1 e 4 de posição, e alugar o intervalo de 3 até 6.

Mário está muito ocupado com seus clientes e pediu que você fizesse um programa para determinar o número mínimo de trocas necessário para satisfazer o pedido de locação de um novo cliente.

\Entrada%
A primeira linha da entrada contém dois números inteiros $N$ e $L$ \mbox{($1 \leq N \leq L \leq 10^5$)}, indicando quantos armários são necessários para acomodar o pedido de locação do novo cliente e quantos armários estão
disponíveis, respectivamente. A segunda linha contém $L$ inteiros distintos $X_i $, em ordem crescente, ($1 \leq X_1 < X_2 < \dots < X_L \leq 10^9$), indicando as posições dos armários disponíveis.

\Saida%
Seu programa deve produzir uma única linha, contendo um único número inteiro, indicando o número mínimo de trocas que Mário precisa efetuar para satisfazer o pedido do novo cliente (ou seja, ter $N$ armários consecutivos disponíveis).

\Exemplos{A_1,A_2,A_3}%
